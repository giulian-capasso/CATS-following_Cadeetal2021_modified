
\documentclass[12pt]{article}
\usepackage[a4paper, margin=1in]{geometry}
\usepackage{graphicx}
\usepackage{booktabs}
\usepackage{amsmath}
\usepackage{hyperref}
\usepackage{parskip}
\usepackage{fancyhdr}
\pagestyle{fancy}
\fancyhf{}
\rhead{Time Step Report}
\lhead{Giulian}
\rfoot{Page \thepage}

\title{\small{Following CADEet al 2021 > skippeddata = find(1.5*1/fs);
    if ~isempty(badrows) || ~isempty(skippeddata)
        if ~isempty(skippeddata)
            disp(['Results in ' num2str(length(skippeddata)) ' gaps in downsampled data.  Should be fixed during prh process.']);
        else
            disp('No gaps found in downsampled data');}}
%\author{Giulian}
%\date{May 19, 2025}

\begin{document}
\maketitle

\section*{\small{1. Overview of the problem}}
Analysis of the continuity of time steps in the CATS DAIRY DATA of the first deployment in Sicily 01-07-2024. Since we had the downsampling error, the goal was to verify whether the time intervals between the samples are consistent and to identify any anomalies or irregularities. 

\section*{\small{2. Visual Inspection}}
The figure below displays the time difference between consecutive timestamps. Regular vertical patterns indicate multiple, repeating sampling intervals, suggesting nonuniform sampling.

\begin{center}
\includegraphics[width=0.8\textwidth]{timeseries_plot.png}
\end{center}

\section*{\small{3. Unique Time Steps Detected}}
The following unique time steps (in seconds) were identified in the data set:

\begin{verbatim}
0.00498199462890625
0.0049896240234375
0.00499725341796875
0.0050048828125
0.00501251220703125
0.0124893188476562
0.0124969482421875
0.0125045776367188
0.01251220703125
0.0125198364257812
0.0174789428710938
0.017486572265625
0.0174942016601562
0.0175018310546875
0.0175094604492188
0.01751708984375
0.0224838256835938
0.022491455078125
0.0224990844726562
0.0225067138671875
0.0225143432617188
0.0274810791015625
0.0274887084960938
0.027496337890625
0.0275039672851562
0.0275115966796875
\end{verbatim}

\section*{4. Approximate Grouping by Sampling Rate}
\begin{center}
\begin{tabular}{@{}lll@{}}
\toprule
\textbf{Approx. dt (s)} & \textbf{Rate (Hz)} & \textbf{Likely Source} \\
\midrule
0.005 & 200     & High-frequency sensor (e.g., accelerometer) \\
0.0125 & 80      & Mid-rate sensor (e.g., magnetometer)        \\
0.0175 & 57.14   & Primary sampling rate (assumed main signal) \\
0.0225 & 44.44   & Low-frequency or drift                      \\
0.0275 & 36.36   & Possible logger issue or buffering delay    \\
\bottomrule
\end{tabular}
\end{center}

\section*{5. Implications}
\begin{itemize}
  \item Non-uniform sampling can distort results from spectral or kinematic analysis.
  \item Tools assuming regular sampling may misinterpret temporal dynamics.
  \item Multiple sensors may not be time-synchronized.
\end{itemize}

\section*{6. Possible Solutions}
\subsection*{Option A: Interpolation to Uniform Sampling}
\begin{verbatim}
t_uniform = t_sec(1):1/57.14:t_sec(end);
x_uniform = interp1(t_sec, x, t_uniform, 'linear');
\end{verbatim}

\subsection*{Option B: Filtering the Main Sampling Rate}
\begin{verbatim}
rounded_dt = round(dt, 4);
idx_main = rounded_dt == 0.0175;
x_main = x(idx_main);
t_main = t_sec(idx_main);
\end{verbatim}

\section*{7. Next Steps}
We recommend standardizing the data based on the most frequent time step or resampling it at a known uniform rate. A custom MATLAB function can be provided to automate this task.


\end{document}
